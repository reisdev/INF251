\documentclass[tikz,12pt]{article}
\usepackage{karnaugh-map}
\graphicspath{./}

\begin{document}

\begin{center}

\author{Matheus dos Reis <matheusdrdj@gmail.com>}

\begin{center}
    \begin{figure}
        \centering
        \includegraphics[scale=0.2]{./doc/logoufv.png}
    \end{figure}
    \centering
    \vfill
    \fontsize{32pt}{1.5} 
    \selectfont Trabalho 2
    \paragraph{}
    \selectfont INF 251
    \vfill
    \fontsize{14pt}{1.5}
    \selectfont MATHEUS DOS REIS DE JESUS
    \paragraph{}
    \selectfont ES89369
    \paragraph{}
    \selectfont VIÇOSA - MG
\end{center}


\newpage

\begin{tikzpicture}[scale=0.2]
\tikzstyle{every node}+=[inner sep=0pt]
\draw [black] (30.8,-13) circle (3);
\draw (30.8,-13) node {$q2$};
\draw [black] (53.5,-31.7) circle (3);
\draw (53.5,-31.7) node {$q3$};
\draw [black] (30.8,-31.1) circle (3);
\draw (30.8,-31.1) node {$q5$};
\draw [black] (30.8,-48.6) circle (3);
\draw (30.8,-48.6) node {$q6$};
\draw [black] (9.7,-31.1) circle (3);
\draw (9.7,-31.1) node {$q4$};
\draw [black] (30.8,-16) -- (30.8,-28.1);
\fill [black] (30.8,-28.1) -- (31.3,-27.3) -- (30.3,-27.3);
\draw (30.3,-22.05) node [left] {$0\mbox{ }|\mbox{ }1$};
\draw [black] (33.8,-31.18) -- (50.5,-31.62);
\fill [black] (50.5,-31.62) -- (49.71,-31.1) -- (49.69,-32.1);
\draw (42.14,-31.93) node [below] {$1$};
\draw [black] (51.09,-33.49) -- (33.21,-46.81);
\fill [black] (33.21,-46.81) -- (34.15,-46.73) -- (33.55,-45.93);
\draw (39.95,-39.65) node [above] {$0\mbox{ }|\mbox{ }1$};
\draw [black] (30.8,-34.1) -- (30.8,-45.6);
\fill [black] (30.8,-45.6) -- (31.3,-44.8) -- (30.3,-44.8);
\draw (30.3,-39.85) node [left] {$0$};
\draw [black] (28.49,-46.68) -- (12.01,-33.02);
\fill [black] (12.01,-33.02) -- (12.31,-33.91) -- (12.94,-33.14);
\draw (22.46,-39.36) node [above] {$0\mbox{ }|\mbox{ }1$};
\draw [black] (11.98,-29.15) -- (28.52,-14.95);
\fill [black] (28.52,-14.95) -- (27.59,-15.09) -- (28.24,-15.85);
\draw (22.46,-22.54) node [below] {$0\mbox{ }|\mbox{ }1$};
\end{tikzpicture}
\end{center}

\newpage

Abaixo, os mapas de karnaugh para as equações do próximo estado

\paragraph{}

\begin{karnaugh-map}[4][4]
    \manualterms{X,0,0,1,0,X,X,X,X,X,1,0,0,0,X,X}
    \implicant{14}{10}
    \implicant{3}{7}
\end{karnaugh-map}

\begin{karnaugh-map}[4][4]
    \manualterms{X,1,1,0,0,X,X,X,X,X,0,0,1,1,X,X}
    \implicant{12}{14}
    \implicant{2}{6}
    \implicant{0}{1}
\end{karnaugh-map}

\begin{karnaugh-map}[4][4]
    \manualterms{X,0,1,0,1,X,X,X,X,X,1,1,0,1,X,X}
    \implicant{8}{10}
    \implicant{5}{15}
    \implicant{2}{10}
\end{karnaugh-map}

\paragraph{}

Abaixo, as equações para cada bit do próximo estado:

\[ p[2] = E_{1}E_{0}\bar{A} \]
\[ p[1] = \bar{E_{2}}\bar{E_{1}}\bar{A} + E_{1}\bar{E_{0}}\bar{A} + E_{2}\bar{A}\]
\[ p[0] = E_{2}E_{0} + E_{1}\bar{E_{0}} + E_{2}A    \]
\end{document}
    